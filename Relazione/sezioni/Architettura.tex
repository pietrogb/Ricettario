\section{Architettura}{
	Il layout è stato strutturato allo scopo di rendere il sito fruibile indipendentemente dal dispositivo, definendo:
	\begin{itemize}\itemsep1pt
		\item layout per dispositivi desktop (suddiviso a seconda del browser)
		\item layout per dispositivi mobili
		\item layout di stampa
	\end{itemize} 
	
	\subsection{Progettazione layout}{
		Si è deciso di utilizzare un layout di tipo responsive a singola colonna principale, facendolo adattare alle dimensioni dello schermo, impostando un limite sulla larghezza dello schermo per passare al layout per dispositivi mobili.
	}
	\subsection{Sviluppo layout}{
		Nell'immagine che segue viene mostrata la struttura data ai vari blocchi \textit{div} che compongono il sito, in cui sono contenute le informazioni divise per area tematica le informazioni.
	\\
	Il layout fluido orizzontalmente si adatta in base alla larghezza dello schermo, con l'obiettivo di far sviluppare verso il basso il sito.
		\begin{figure}[H]
			\includegraphics{\docsImg Schemasito.png}
			\caption{Schema del sito}
			\label{Schema del sito}
		\end{figure}
		
		Il foglio di stile standard viene utilizzato sui browser per computer desktop e portatili, fino a che la larghezza dello schermo rimane maggiore di 650 px: al di sotto si passa automaticamante ad usare il CSS destinato al mobile.
		\\
		Analizzando il sito spostandosi dall'alto verso il basso, segue che:
		\begin{itemize}\itemsep1pt
			\item Il div \textbf{header} che informa l'utente su ciò che sta visitando; comprende:
			\begin{itemize}\itemsep1pt
				\item Titolo e logo; nella pagina relativa alle vendite è presente inoltre il pulsante d'accesso amministratore che mostra un menù a scomparsa da cui poter effettuare il login;
				\item Il div \textbf{breadcrumbs} che aiuta l'utente ad identificare la posizione in cui si trova all'interno del sito, rispetto alla homepage; al suo interno è presente anche la barra di ricerca con cui trovare i prodotti all'interno del sito;
				\item Il menù utente, identificato con il div \textbf{menu}, presenta 4 scelte: \textit{Home}, \textit{Realizzazioni}, \textit{Vendita} e \textit{Contattaci}. Viene evidenziata la posizione corrente. Situato centralmente alla pagine e si estende orizzontalmente.
			\end{itemize}
			\item Segue il contenuto vero e proprio della pagina, inserito nel div \textbf{content}: ha il compito di esporre le informazioni che si stanno trattando;
			\item Alla fine della pagina, il \textbf{footer} esposto con un layout a 3 colonne. Al cui interno è presente:
			\begin{itemize}\itemsep1pt
				\item un piccolo logo dell'azienda;
				\item collegamenti alle pagine del sito;
				\item i principali riferimenti all'azienda;
				\item motto aziendale.
			\end{itemize}
		\end{itemize}
		}
	\subsection{Layout per dispositivi mobili}{
		Il layout per dispositivi mobili è stato sviluppato in modo da favorire l'incolonnamento degli elementi, rimuovendo quanto possibili margini e padding, sfruttando al meglio l'area disponibile; sono state infine ridimensionate le immagini presenti.
	}
	\subsection{Layout di stampa}{
		Nel layout di stampa sono stati tolti gli elementi che non portavano informazioni significative; i contenuti sono stati privati dei colori; è stato rimosso il menù; nella pagina delle realizzazioni è stata messa in risalto l'immagine selezionata; nella pagina relativa alla vendite si è fatto in modo da non spezzare su più pagine le informazioni relative al singolo articolo all'interno del menù.
	}
}
