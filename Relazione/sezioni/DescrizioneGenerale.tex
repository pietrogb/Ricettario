\section{Descrizione Generale}{
	\subsection{Il sito}{
		Il sito intende modellare un ricettario on-line in cui sia possibile inserire e consultare ricete. Utilizza un'interfaccia web in modo da renderne possibile l'utilizzo su una gran numero di dispositivi, ed un database MySql in cui sono memorizzate le ricette.
		\\
		In più, è stata implementata una politica per il controllo degli accessi, che consente a chiunque di visualizzare le ricette, ma permette l'uso delle funzionalità che effettuano modifiche ai soli utenti autenticati.
		\\
		Le ricette sono organizzate per categorie, inoltre è semplice crearne di nuove. Gli ingredienti sono aggiunti nel database on-the-fly, mano a mano che vengono aggiunti alle ricette. Le unità di misura (grammi, cucchiai), sono memorizzati anch'essi nel database e possono essere aggiornati o modificati.
		\\
		È stata inclusa una funzionalità di ricerca che permette agli utenti di effettuare ricerce utilizzando un ampio spettro di criteri, incluse parole trovate (o non trovate) nelle istruzioni, il nome della ricetta, categoria o lista di ingredienti.

	}
	\subsection{Caratteristiche degli utenti}{
		Il sito intende rivolgersi ad un pubblico generico, non espressamente esperto d'informatica. 
		Ambisce ad essere quanto più simile ad un quaderno che può essere riempito a poco a poco e consultato alla bisogna.

		% Il sito è rivolto ad un pubblico generico, all'interno del quale possiamo individuare le seguenti categorie:
		% \begin{description}\itemsep1pt
		% 	\item[Categoria di utenti:] privati e piccole aziende
		% 	\begin{description}\itemsep1pt
		% 		\item[Funzionalità:] Informarsi sugli attrezzi per il giardinaggio prima di procedere all'acquisto presso la sede dell'azienda. Contattare tramite form dedicata l'azienda.
		% 		\item[Termini generali:] Non eccessivamente distante dal punto vendita, in un raggio di circa 70 Km.
		% 	\end{description}
		% 	\item[Categoria di utenti:] amministratori
		% 	\begin{description}\itemsep1pt
		% 		\item[Funzionalità:] area riservata da cui aggiungere, rimuovere o aggiornare prodotti disponibili;
		% 	\end{description}
		% \end{description}
	}
	\subsection{Vincoli generali}{
		\begin{itemize}\itemsep1pt
			\item Il sito dev'essere accessibile da parte di categorie d'utenti diversificate ed utilizzando dispositivi diversi compresi smartphones e tablet;
			\item L'interfaccia grafica dev'essere semplice;
			\item Tutte le categorie e le unità di misura devono essere modificabili;
			\item Deve essere garantita la sicurezza, in modo da impedire modifiche esterne indesiderate;
			\item Tutte le pagine devono essere basate su di un template, in modo da rendere semplice la modifica;
			\item Il sito dev'essere visitabile tramite i seguenti browser: 
				\begin{itemize}
					\item Firefox 3.6;
					\item Internet Explorer dalla versione 7 alla versione 11; Edge 13;
					\item Chrome 14;
					\item Opera 12.16;
					\item Safari 9.
				\end{itemize}
			\item Separazione tra struttura, presentazione, comportamento;
			\item Conformità agli standard W3C per XHTML, CSS, JS;
			\item Sito comprensibile da screen-reader.
		\end{itemize}
	}
	\subsection{Requisiti}{
		Di seguito sono presentati i requisiti emersi dall'analisi iniziale e quelli che si sono aggiunti nel corso dello svolgimento del progetto. Ciascuno è identificato da un numero progressivo per semplificarne l'individuazione successiva.\\
		\newcounter{magicrownumbers}
		\newcommand\rownumber{\stepcounter{magicrownumbers}\arabic{magicrownumbers}}
		\begin{table}[h]
			\centering
			\begin{tabular}{|p{\dimexpr 0.15\linewidth-2\tabcolsep}|p{\dimexpr 0.8\linewidth-2\tabcolsep}|}
				\hline
			 	\textbf{ID Req.} & \textbf{Descrizione}\\
				\hline
				\centering \rownumber	&	Il sito dev'essere visualizzabile sui browser elencati all'interno di "Vincoli generali"\\
				\hline
				\centering \rownumber	&	Il sito dev'essere accessibile indipendentemente dalla grandezza dello schermo del dispositivo\\
				\hline
				\centering \rownumber	&	Il sito dev'essere fruibile anche senza richiedere un foglio di stile\\
				\hline
				\centering \rownumber	&	Le figure significative dovranno essere comprensive di un attributo alt per favorire l'accesso ad utenti non vedenti\\
				\hline
				\centering \rownumber	&	Ad i tag quali <input> e <textarea> devono essere associati tabindex e accesskey\\
				\hline
				\centering \rownumber	&	Le gradazioni di colori non devono risultare sgradevoli o di intralcio a persone affette da daltonismo\\
				\hline
				\centering \rownumber	&	Il layout deve risultare fluido nel ridimensionamento del carattere tramite i tasti Ctrl + e Ctrl -\\
				\hline
				\centering \rownumber	&	Il sito deve essere validato per la parte di XHTML2.0, CSS3 e secondo gli standard WAI\\
				\hline
			\end{tabular}
			\label{tab:requisiti}
			\caption{Elenco dei requisiti}
		\end{table}
	}
}
