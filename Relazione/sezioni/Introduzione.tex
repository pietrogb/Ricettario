\section{Introduzione}{
	\textbf{\ggt} è un'applicazione che modella e rende disponibile on-line un ricettario\\
	Lo scopo principale del sito è di gestire le ricette inserite dagli utenti; le operazioni principali sono la registrazione di un nuovo utente, la creazione, modifica e cancellazione d'una ricetta.
	
    \subsection{Riferimenti}{
		Per la progettazione del sito, si è fatto riferimento alle seguenti normative e specifiche:
		\begin{itemize}\itemsep0.5pt
%			\item Legge Stanca \url{www.agid.gov.it/agenda-digitale/pubblica-amministrazione/accessibilita};
			\item Legislazione e linee guida per accessibilità dei siti web istituzionali, 2011 \url{http://www.math.unipd.it/~artico/direttiva.htm}
			\item Specifiche Web Accessibility Initiative (WAI)  \url{http://www.w3.org/WAI};
			\item Specifiche Web Content Accessibility Guidelines (WCAG) 2.0 \url{www.w3.org/TR/WCAG20/};
			\item Specifiche Sezione 508 \url{https://www.section508.gov/};
			\item Ruota dei colori accessibile \url{http://colorfilter.wickline.org/};
			\item Slides del corso: \url{http://docenti.math.unipd.it/gaggi/tecweb/materiale.html}.
			\item Risorse di WebAIM, Web Accessibility In Mind: \url{http://webaim.org/resources}
		\end{itemize}
    }
}
